\documentclass[a4paper]{article}
%\usepackage[top=30pt,bottom=40pt,left=48pt,right=46pt]{geometry}

\usepackage{amsmath}
\usepackage{amssymb}
\usepackage{enumitem}
\usepackage{csquotes}
\usepackage{graphicx}
\usepackage[ngerman]{babel}
\usepackage{setspace}

\title{95 Gründe, warum eine Online-Abgabe nötig ist}
\author{Studenten}

\setcounter{secnumdepth}{0}


\onehalfspacing

\begin{document}

\maketitle

\begin{enumerate}

	\item Wir befinden uns in einem digitalen Zeitalter, eine schriftliche Abgabe wäre nur ein Dämpfer für den Fortschritt
	\item Man muss keine Zeit aufwenden die Blätter zu den einzelnen Tutorien zuzuordnen, das zeugt von einer besseren Organisation
	\item Alles läuft über Ilias, da gibt es keine Viren
	\item Die Tutoren wären sehr erfreut darüber
	\item Die Studenten wären sehr erfreut darüber
	\item Jeder, der die Abgabe schriftlich verfasst, macht das gezwungenermaßen
	\item Sehr viele Vorlesungen sind bei einer Online-Abgabe geblieben
	\item Es ist auch möglich Online- und Offline-Abgabe gleichzeitig zu machen (LA 1 im WS 21/22)
	\item Eine digitale Abgabe ermöglicht eine leichteren Weg für Leute mit Laptop/Tablet etwas abzugeben
	\item Da die Rückmeldung auch online ablaufen würde, ist für Studenten mit einer früheren Rückmeldung zu rechnen, womit sie sich auf das Tutorium vorbereiten können
	\item Da alle Abgaben online Ablaufen, haben beide Parteien jeweils die Abgabe und Rückmeldung jederzeit zur Verfügung, was zu einer besseren Dokumentation führt
	\item Als PDF abzugeben spart Papier, das ist gut für die Umwelt
	\item Auch bei einer Online-Abgabe lernen Studenten rechtzeitig Dokumente abzugeben, es herrscht eine punktgenaue Abgabezeit
	\item Ein Riesenbriefkasten ist nicht nötig
	\item Die Übungsleiter und Tutoren müssen nicht mehr Gewicht mit sich rumschleppen, das ist gut für die Gesundheit
	\item Eine Online-Abgabe ist ein Erkennungszeichen für eine \textbf{moderne} Eliteuni
	\item Papier kann bei Wind wegfliegen
	\item Der Computer und damit die PDF kann nicht bei Wind wegfliegen
		
	% Dieser Punkt als letztes	
	\item Und zu guter Letzt: Mit Online-Abgabe gäbe es diese Liste nicht
\end{enumerate}

\end{document}